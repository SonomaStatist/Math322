\documentclass[a4paper,12pt]{report}

\usepackage{../ssumath}

\begin{document}

\mktitle{Math 322 -- Linear Algebra}{Homework 2}{Amandeep Gill}

\proof{1}{
	If $V,W$ are vector spaces over a field $F$, with $V_1$ and $W_1$ respective subspaces of $V$ and $W$, and with a linear function $T:V \rightarrow W$, then $T(V_1)$ is a subspace of $W$ and $\{\vec{x} : T(\vec{x}) \in W_1\}$ is a subspace of $V$.
}{
	Let $V,W$ be vector spaces over a field $F$, with $V_1$ and $W_1$ respective subspaces of $V$ and $W$, and let $T:V \rightarrow W$ be a linear function.
	
	\item[(a)] $T(V_1)$ is a subspace of $W$.
	
	Since $V_1$ is a subspace of $V$, for all $\vec{x} \in V_1$ there exists $\vec{x_1},\vec{x_2} \in V_1$ and $c \in F$ such that $\vec{x} = c \vec{x_1} + \vec{x_2}$. Using the definition of linearity of $T$, $T(\vec{x}) = T(c \vec{x_1} + \vec{x_2}) = c T(\vec{x_1}) + T(\vec{x_2})$. Therefore $T(V_1)$ is closed for vector addition and scalar multiplication, and is thus a  subspace of $W$.
	
	\item[(b)] $\{\vec{x} : T(\vec{x}) \in W_1\}$ is a subspace of $V$.
	
	Let $V_T = \{\vec{x} : T(\vec{x}) \in W_1\}$. For all $\vec{x} \in V_T$, there exists $\vec{x_1},\vec{x_2} \in V_T$ and $c \in F$ such that $T(\vec{x}) = c T(\vec{x_1}) + T(\vec{x_2})$ as $T(V_T)$ is a subspace of $W$. Since $T$ is linear, $T(\vec{x}) = T(c \vec{x_1} + \vec{x_2})$ and $\vec{x} =  c \vec{x_1} + \vec{x_2}$. $V_T$ is thus closed under scalar multiplication and vector addition and is a subspace of $V$.
}

\proof{2}{
	If $B \in M_{n \times n}(F)$ such that $B$ is an invertible matrix, then the function $\Phi : M_{n \times n}(F) \rightarrow M_{n \times n}(F)$ such that $\Phi (A) = B^{-1}AB$, then $\Phi$ is an isomorphism.
}{
	Let $c \in F$ and $A_1,A_2 \in M_{n \times n}(F)$ such that for all $A \in M_{n \times n}$, we have $A = cA_1 + A_2$, then $\Phi (cA_1 + A_2) = B^{-1}(cA_1 + A_2)B$. By the distributive and scalar multiplicative laws for $n \times n$ matrices, we have $\Phi (cA_1 + A_2) = cB^{-1}A_1B + B^{-1}A_2B = c\Phi(A_1) + \Phi(A_2)$. So $\Phi$ is linear. Further, for all $A$ if $\Phi(A) = 0_n$ then $B^{-1}AB = 0_n$ and $B(B^{-1}AB)B^{-1} = 0_n$, so by associativity, $A = 0_n$ and null($\Phi$) = $\{0_n\}$. Therefore $\Phi$ is an isomorphism.
}

\pagebreak

\problem{3}{
	Let $V,W$ be vector spaces over $F$ and $T,U:V \rightarrow W$ be linear.
}{
	\subproof{(a)}{
		$R(T + U) = R(T) + R(U)$
	}{
		Using the definition of function addition, $R(T + U) = (T + U)(V)$. By the same, $(T + U)(V) = T(V) + U(V)$. Since $R(T) = T(V)$ and $R(U) = U(V)$. Thus $R(T + U) = R(T) + R(U)$.
	}
	\subproof{(b)}{
		If $\func{dim}{W} \in \bb{N}$ then $\func{rank}{T+U} \leqslant \func{rank}{T} + \func{rank}{U}$
	}{
		By part (a), $R(T + U) \subseteq R(T) + R(U)$. $\func{rank}{T + U} = \func{dim}{R(T + U)}$. Using the Dimension Theorem for finite vector spaces, we have that $\func{dim}{R(T + U)} = \func{dim}{R(T)} + \func{dim}{R(U)} - \func{dim}{R(T) \cap R(U)}$. Since $\func{dim}{R(T) \cap R(U)} \geqslant 0$, $\func{rank}{T + U} \leqslant \func{dim}{R(T)} + \func{dim}{R(U)} = \func{rank}{T} + \func{rank}{U}$.
	}
	\subproof{(c)}{
		Deduce that $\func{rank}{A+B} \leqslant \func{rank}{A} + \func{rank}{B}$ for $A,B \in M_{m \times n}(F)$
	}{
		Let $\func{dim}{V} = n$, $\func{dim}{W} = m$ and $A,B \in M_{m \times n}(F)$,  such that $L_A = T$, and $L_B = U$. Then $\func{rank}{T} = \func{rank}{L_A} = \func{rank}{A}$ and $\func{rank}{U} = \func{rank}{L_B} = \func{rank}{B}$. Using the result obtained from part (b),  $\func{rank}{T + U} \leqslant \func{rank}{T} + \func{rank}{U} = \func{rank}{A} + \func{rank}{B}$. Since $\func{rank}{A + B} = \func{rank}{T + U}$, $\func{rank}{A + B} \leqslant \func{rank}{A} + \func{rank}{B}$.
	}
}

\end{document}